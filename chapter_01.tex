\documentclass[11pt]{article}
\usepackage{amsmath,amssymb,amsthm}
\usepackage{graphicx}
\usepackage[margin=1in]{geometry}
\usepackage{fancyhdr}
\setlength{\parindent}{0pt}
\setlength{\parskip}{5pt plus 1pt}
\setlength{\headheight}{13.6pt}
\newcommand\question[2]{\vspace{.25in}\hrule\textbf{#1: #2}\vspace{.5em}\hrule\vspace{.10in}}
\renewcommand\part[1]{\vspace{.10in}\textbf{(#1)}}
\newcommand{\E}{\mathbb{E}}
\newcommand{\V}{\mathbb{V}}
\newcommand{\dx}{\mathrm{d}x}
\newcommand{\dy}{\mathrm{d}y}
\pagestyle{fancyplain}
\lhead{\textbf{\NAME}}
\chead{\textbf{Solutions: \HWNUM}}
\rhead{\today}
\begin{document}\raggedright
%Section A==============Change the values below to match your information==================
\newcommand\NAME{Omar Sosa Rodriguez}
\newcommand\HWNUM{Ch. 1 - Preliminaries}

\question{1}{Daycount and frequency} 

\part{a} Using either a semi-bond or annual-money, the accrued value after one year (365 days) should be the same. Therefore
\begin{align}
    (1 + y_{SB}\frac{180}{360})^2 &= (1 + y_{AM} \frac{365}{360}) \nonumber\\
    \implies y_{AM} &= \frac{360}{365}\left[(1 + \frac{y_{SB}}{2})^2 - 1\right]
\end{align}

\part{b} We simply need to plug in the different values in the above result:
\begin{align}
    y_{AM}\bigg\rvert_{y_{SB} = 0.05} \approx 0.04993 \\
    y_{AM}\bigg\rvert_{y_{SB} = 0.06} \approx 0.06007 \\
    y_{AM}\bigg\rvert_{y_{SB} = 0.07} \approx 0.07024    
\end{align}
We see that when $y_{SB}\in\left[0.05, 0.07\right]$ then $y_{AM}\in\left[0.04993, 0.07024\right]$ so we conclude that $y_{AM}$ has higher standard deviation.

\part{c}From part \textbf{(a)} we have that
\begin{align}
    \frac{\mathrm{d}y_{AM}}{y_{AM}} &= \frac{360}{365 y_{AM}}\left[\left(1 + \frac{y_{SB}}{2}\right)\mathrm{d}y_{SB}\right] \nonumber\\
                                    &= \frac{\left(1 + \frac{y_{SB}}{2}\right)\mathrm{d}y_{SB}}{\left(1 + \frac{y_{SB}}{2}\right)^2 - 1}\nonumber\\
                                    &= \frac{\left(1 + \frac{y_{SB}}{2}\right)}{\left(1 + \frac{y_{SB}}{4}\right)}\frac{\mathrm{d}y_{SB}}{y_{SB}} \\
\end{align}
Then, at $y_{SB} = 0.06$, the ratio of the volatilities is $1.03/1.025 \approx 1.0049$.

\question{2}{Simple interest} Let $r, r_A, r_Q$ and $r_C$ be the simple, annually compouded, quarterly compounded, and continuous interest rates respectively. After $T$ years, they should yield the same accrued values and therefore
\begin{align}
    1 + r T = (1 + r_A)^T = (1 +\frac{r_Q}{4})^{4T} = e^{r_C T} .
\end{align}
So that,
\begin{align}
    r_A &= (1 + r T)^{1/T} - 1 \label{eq:annual}\\
    r_Q &= 4\left[(1 + r T)^{1/4T} - 1\right]\\
    r_C &= \frac{1}{T}\log(1 + r T)
\end{align}
Finally, by plugging in $r=0.05$ and $T=10$, we get $r_A \approx 0.0413$, $r_Q \approx 0.0407$ and $r_C = 0.0405$.
\part{b} We take the limit as $T\to\infty$ of equation \eqref{eq:annual}:
\begin{align}
    r_A \to \lim_{T\to\infty} (1 + r T)^{1/T} - 1 &= \lim_{T\to\infty} e^{\frac{1}{T}\log(1 + r T)} - 1\nonumber \\
                                          &= e^{\lim_{T\to\infty}\frac{1}{T}\log(1 + r T)} - 1\nonumber \\
                                          &= e^{\lim_{T\to\infty}\frac{r}{1 + rT}} - 1\nonumber \\
                                          &= e^{0} - 1\nonumber \\
                                          &= 0
\end{align}

\question{3}{Non-standard annuity}
\part{a} Let today be denoted by $t=0$ and the maturity date by $T$. The phrasing of the problem is a bit vague, but let's assume that the payments are made each year. Then, the value today is
\begin{align}
    V &= \sum_{n=1}^{T} \frac{n}{(1 + r)^n} \nonumber\\
    \implies \frac{V_T}{(1 + r)} &= \sum_{n=1}^{T} \frac{n}{(1 + r)^{n+1}} \nonumber\\
                                 &= \sum_{n=1}^{T} \frac{n+1}{(1 + r)^{n+1}} - \sum_{n=1}^{T} \frac{1}{(1 + r)^{n+1}} \nonumber\\
                                 &= \sum_{n=2}^{T + 1} \frac{k}{(1 + r)^{k}} -\frac{1}{r(1 + r)}\left(1 - \frac{1}{(1 + r)^T}\right) \nonumber \\
                                 &= -\frac{1}{1 + r} + \frac{T + 1}{(1 + r)^{T+1}} + \sum_{n=1}^{T} \frac{k}{(1 + r)^{k}} -\frac{1}{r(1 + r)}\left(1 - \frac{1}{(1 + r)^T}\right) \nonumber\\
                                 &= -\frac{1}{1 + r} + \frac{T + 1}{(1 + r)^{T+1}} + V - \frac{1}{r(1 + r)}\left(1 - \frac{1}{(1 + r)^T}\right) \nonumber \\
   \implies V &= \frac{1}{r}\left(1 - \frac{T + 1}{(1 + r)^T}\right) + \frac{1}{r^2}\left(1 - \frac{1}{(1 + r)^T}\right)
\end{align}
\part{b} Taking the limit of the above as $T\to\infty$ we get that $V\to (r + 1)/r^2$.

\end{document}